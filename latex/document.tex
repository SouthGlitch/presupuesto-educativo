\documentclass[12pt]{article}
\usepackage[utf8]{inputenc}
% \usepackage{natbib}
\usepackage[natbibapa]{apacite}
\usepackage{chronology}
% \usepackage{hyphenat}

\title{
    La educación en el presupuesto nacional (2016-2021)
}
\author{Nahuel Arturo Rabey}
\date{14 de febrero, año 2022}

\begin{document}

\begin{titlepage}
    \maketitle
\end{titlepage}

\section{Introducción}

El desarrollo productivo de un país, y las consecuentes políticas públicas que lo acompañan, no existen aisladas del proyecto político que llevan las fuerzas gobernantes. Las disputas en el plano económico tienen su reflejo en el plano político: la UIA contra el cepo al dolar \cite{tisalta} o Martín Migoya - CEO de Globant - oponiéndose a la suspensión de las S.A.S\footnote{Sociedades por Acciones Simplificadas} \cite{globant}

La educación no queda ajena al terreno de estos debates. La coordinación de la educación con la producción, reflejada en las leyes educativas, los gobiernos universitarios y, por su puesto, los objetivos presupuestarios, requiere un abordaje integral del problema. Si un proyecto de gobierno considera el desarrollo industrial cómo principal objetivo, en el caso del siglo XX con la sustitución de importaciones, tendrá un cambio en el presupuesto educativo, no sólo en cantidad sino en composición. Ejemplo de ello es la Universidad Obrera Nacional, creada en 1949 durante el gobierno del General Perón.

Acompañado de un proyecto están quienes lo llevan delante. El estudiantado y la docencia no forman una masa uniforme, con un desarrollo de ideas lineal, muy por el contrario surgen discusiones sobre cual dirección tomar, y en particular, discusiones que afectan el mundo del trabajo, el productivo, dónde el resultado educativo toma su valor, en el sentido más mercantil de la palabra, porque sin demanda de producto no hay quién lo compre.

Las reformas educativas son claros reflejos de los objetivos productivos: la ley federal de educación durante el menemismo, que dejaba abierta el arancelamiento de la educación superior, la Ley de Educación Nacional que definía un 6\% del PBI para la educación \cite{cipec}, la UNICABA o la secundaria del futuro sin docentes del macrismo (buscar fuentes), etc. Cada una de estas acciones vio su reacción en la comunidad educativa, las asambleas, movilizaciones, plenarias, pusieron - potencialmente o efectivamente - en discusión la política educativa para toda la población, y a través de ella la política económica y productiva <<buscar ejemplos>>

Es con esta triada, gremial, productiva y presupuestaria (recursos disponi-bles) que deben entenderse las decisiones de gobierno, y a través de ella medir sus aciertos. Con principal eje en el análisis presupuestario, su evolución a lo largo de los últimos ocho años, es que el presente proyecto de investigación parte de las siguiente hipótesis: los cambios en la estructura del presupuesto nacional no aumentaron ni redujeron el presupuesto destinado al sistema educativo, cómo consecuencia, y a agravado por la actual pandemia, el debate sobre el futuro de la economía pasará nodalmente cómo un debate del sistema educativo y, a su vez, tomará la forma de nuevas reformas en un futuro.

\subsection*{Inflación y metodología}
Nótese, previo a cualquier análisis presupuestario, la necesidad imperante de una medida común para evaluar cada ítem, sección, de una forma tal que sea comprensible para el lector de hoy en día. No son equivalentes 500 AR\$ en el 2022 que aquellos usados en el 2011. Es por ello que todo valor monetario presente en el presupuesto, noticias y artículos citados ha sido convertido a pesos constantes diciembre 2021.

Esta tarea, en sí misma, ya presenta dos dificultades principales. La primera, el Indice de Precios al Consumidor, antes del año 2016, no fue tomado desde una perspectiva federal, sino muy por el contrario reducido a las áreas urbanas.

Otra de sus dificultades radica en el cambio de bases que existió durante el gobierno del FPV, dos en doce años. Estos son IPC base abril 2008 y el IPC julio 2013 agosto 2014, que sigue hasta el primer mes de 2015.

A partir de esto momento, el nuevo gobierno de Cambiemos declara la emergencia estadística, razón por la cual deja de medirse el IPC hasta diciembre del 2016.

Estas irregularidades, particularmente en el período 2014-2015, llevan a que no haya una medida única, estandarizada por el INDEC, para poder utilizar. Por lo tanto se resuelve empalmar cuatro series de datos, que permitirán el cálculo de pesos constantes a usar en la investigación:

\begin{chronology}[1]{2011}{2022}{\textwidth}
    \event[\decimaldate{1}{1}{2011}]{\decimaldate{01}{12}{2013}}{IPC INDEC base 2006}
    \event[\decimaldate{1}{12}{2013}]{\decimaldate{01}{10}{2015}}{IPC INDEC base octubre 13 septiembre 2014}
    \event[\decimaldate{1}{10}{2015}]{\decimaldate{01}{12}{2016}}{IPC CABA base 2012}
    \event[\decimaldate{1}{12}{2016}]{\decimaldate{01}{12}{2021}}{IPC INDEC base diciembre 2016}
\end{chronology}

\tableofcontents


\section{Gobierno de Cambiemos, endeudarse para no imprimir}
\section{El Fondo Monetario Internacional}
\section{Frente de Todos, pandemia y crisis internacional}

\bibliography{./biblio.bib}
\bibliographystyle{apacite}

\end{document}