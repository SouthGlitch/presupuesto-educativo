\documentclass[12pt, a4]{article}
\usepackage[utf8]{inputenc}
\usepackage{chronology}
% \usepackage{hyphenat}

\title{
    La educación en el presupuesto nacional
}
\author{Nahuel Arturo Rabey}
\date{14 de febrero, año 2022}

\begin{document}

\begin{titlepage}
    \maketitle
\end{titlepage}

\tableofcontents

\section{Introducción}
En cualquier Nación naciente la educación tiene un rol fundamental, en particular, en aquellos tiempos dónde los limites fronterizos marcaban un límite arancelario, un plan de desarrollo económico y una cultura, dónde se desarrolla la niñez de todo habitante.

Con el desarrollo industrial el proyecto educativo de una incipiente clase obrera cobra un rol fundamental, se alza la cuestión central, imperante ¿qué necesita del obrero una fábrica para funcionar? La escuela prusiana, con concepción universal, el modelo fabril toma control del educativo, dónde, cómo una línea de producción asemejara, el alumno pasa por grados, años y sectores, que formaran esa masa de trabajadores y jornaleros de la futura nación Germana.

La coordinación de la educación con la producción, reflejada en las leyes educativas, los gobiernos universitarios y, por su puesto, los objetivos presupuestarios, deben tratarse cómo única visión. Si un proyecto de gobierno considera el desarrollo industrial cómo principal objetivo, en el caso del siglo XX con la sustitución de importaciones, tendrá un cambio en el presupuesto educativo, no sólo en cantidad sino en composición. Ejemplo de ello es la Universidad Obrera Nacional, creada en 1949 durante el gobierno del General Perón.

Acompañado de un proyecto están quienes lo llevan delante. El estudiantado y la docencia no forman una masa uniforme, con un desarrollo de ideas lineal, muy por el contrario surgen discusiones sobre cual dirección tomar, y en particular, discusiones que afectan el mundo del trabajo, el productivo, dónde el resultado educativo toma su valor, en el sentido más mercantil de la palabra, porque sin demanda de producto no hay quién lo compre.

Es con esta triada, gremial, productiva y presupuestaria (recursos disponi-bles) que deben entenderse las decisiones de gobierno, y a través de ella medir sus aciertos. Con principal eje en el análisis presupuestario, su evolución a lo largo de los últimos ocho años, es que el presente proyecto de investigación busca responder la siguiente pregunta ¿qué consecuencias tuvo el modelo productivo en el sistema educativo?

\section*{Inflación y metodología}
Nótese, previo a cualquier análisis presupuestario, la necesidad imperante de una medida común para evaluar cada ítem, sección, de una forma tal que sea comprensible para el lector de hoy en día. No son equivalentes 500 AR\$ en el 2022 que aquellos usados en el 2011. Es por ello que todo valor monetario presente en el presupuesto, noticias y artículos citados ha sido convertido a pesos constantes diciembre 2021.

Esta tarea, en sí misma, ya presenta dos dificultades principales. La primera, el Indice de Precios al Consumidor, antes del año 2016, no fue tomado desde una perspectiva federal, sino muy por el contrario reducido a las áreas urbanas.

Otra de sus dificultades radica en el cambio de bases que existió durante el gobierno del FPV, dos en doce años. Estos son IPC base abril 2008 y el IPC julio 2013 agosto 2014, que sigue hasta el primer mes de 2015.

A partir de esto momento, el nuevo gobierno de Cambiemos declara la emergencia estadística, razón por la cual deja de medirse el IPC hasta diciembre del 2016.

Estas irregularidades, particularmente en el período 2014-2015, llevan a que no haya una medida única, estandarizada por el INDEC, para poder utilizar. Por lo tanto se resuelve empalmar cuatro series de datos, que permitirán el cálculo de pesos constantes a usar en la investigación:

\begin{chronology}[1]{2011}{2022}{\textwidth}
    \event[\decimaldate{1}{1}{2011}]{\decimaldate{01}{12}{2013}}{IPC INDEC base 2006}
    \event[\decimaldate{1}{12}{2013}]{\decimaldate{01}{10}{2015}}{IPC INDEC base octubre 13 septiembre 2014}
    \event[\decimaldate{1}{10}{2015}]{\decimaldate{01}{12}{2016}}{IPC CABA base 2012}
    \event[\decimaldate{1}{12}{2016}]{\decimaldate{01}{12}{2021}}{IPC INDEC base diciembre 2016}
\end{chronology}

asdas

\section{Último gobierno del Frente Para la Victoria \\ \large 2011-2015}
\section{Gobierno de Cambiemos \\ \large 2015-2021}
\section{Primeros pasos del Frente de Todos \\ \large 2021}

\end{document}